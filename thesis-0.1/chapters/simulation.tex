\chapter{Results}
\label{chapter:results}
\section{Python/C++ Interface :}
The user interface consists of:
\subsection{C++ Library :} 
\begin{itemize}
\item C++ Class Record : specifies which results should be stored and in which interval.
\item C++ Class Material : contains the characteristics of each material (name, electric permittivity, magnetic permeability, sigma).
\item C++ Class Region : includes the specifications of each region (name, start, end, material index).
\item C++ Class Device : lists the materials and regions applied in the simulation.
\item C++ Class Scenario (string name, double total time, double timestep, List Record records): 
\item C++ Class Result (string name, List double data)
\item C++ Function: List Result simul(Device d, Scenario s)Depending on the Device and Scenario , determine the results (electric field, population on each level..)
\end{itemize}

\subsection{Interface :}
CMakeLists.txt file compiles the different modules (source code: interface file, header file and C++ files) using SWIG and links them to the Tesprogram

\subsection{Python Testprogram :}
The testprogram (Skript and Notebook) allows the setup of Materials, Device, Scenario etc., extracts Metadata from an XML File, calls the C++ Function that calculates the required results, displays them and  stores them in a ddition to the Metadata in a HDF5 File.

\section{Compatibility with Python3 :}
SWIG is compatible with the different versions of Python (Python 2.7 and Python 3.x). To choose a specific version, all that has to be done is to indicate it in the CMakeLists.txt File, which will find the corresponding packages and use them to compile and link.

\section{SWIG on Windows :}
Another approach can be applied to achieve these results on Windows: using a Setup script (setup.py file). The only problem it might cause is structuring the files in the project: the setup file needs all files required to build the program in one directory and it generates data in the same folder.
