\chapter{Results}
\label{chapter:results}
\section{Python/C++ Interface}
The user interface consists of:
\subsection{C++ Library} 
\begin{itemize}
\item C++ Class Record: specifies which results should be stored and in which interval.
\item C++ Class Material: contains the characteristics of each material (name, electric permittivity, magnetic permeability).
\item C++ Class Region: includes the specifications of each region (name, start, end, material index).
\item C++ Class Device: lists the materials and regions applied in the simulation.
\item C++ Class Scenario: contains the total time of the simulation, the timestep and the list of the records.
\item C++ Class Result: gives a vector of values corresponding to the measured result (electric field, elements of the density matrix)
\item C++ Function: Depending on the Device and Scenario, calculates the results.
\end{itemize}

\subsection{Interface}
SWIG can be incorporated in the build system CMake, which can detect the SWIG executable and many of the target language libraries for linking against and knows how to build shared libraries and loadable modules on many different operating systems. Using a single cross platform file (CMakeLists.txt) and two simple commands: cmake and make, CMake generates native build files such as makefiles, nmake files and Visual Studio projects which will invoke SWIG and compile the generated C++ files into  \textunderscore(module\textunderscore name).so (UNIX) or  \textunderscore(module\textunderscore name).pyd (Windows).

\subsection{Python Testprogram}
The testprogram (Skript and Notebook) allows the setup of Materials, Device, Scenario etc., extracts Metadata from an XML File, calls the C++ Function that calculates the required results, displays them and  stores them in addition to the Metadata in a HDF5 File.

\section{Compatibility with Python3}
SWIG is compatible with the different versions of Python (Python 2.7 and Python 3.x). To choose a specific version, all that has to be done is to indicate it in the CMakeLists.txt File, which will find the corresponding packages and use them to compile and link.

\section{SWIG on Windows}
Another approach is available on Windows: building the extension module using a configuration file (conventionally called setup.py), it creates an Extension module object using the source code files generated by swig, in addition to the original C++ source and compiles it into a shared object file or DLL (.pyd on Windows), which can be called in Python testprogram. 


