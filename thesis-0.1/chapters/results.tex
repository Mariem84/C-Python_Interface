\chapter{Results}
\label{chapter:results}
Most of the work was realised on Linux and delivered the required Interface consisting of different parts.
\section{Python/C++ Interface}
\subsection{Source files} 
These include the interface file, needed by SWIG, the header file, declaring the classes and functions as well as the C++ source code.
\begin{itemize}
\item C++ Class Record: specifies which results should be stored and in which interval.
\item C++ Class Material: contains the characteristics of each material (name, electric permittivity, magnetic permeability).
\item C++ Class Region: includes the specifications of each region (name, start, end, material index).
\item C++ Class Device: lists the materials and regions applied in the simulation.
\item C++ Class Scenario: contains the total time of the simulation, the timestep and the list of the records.
\item C++ Class Result: gives a vector of values corresponding to the measured result (electric field, elements of the density matrix)
\item C++ Function: Depending on the device and scenario, calculates the results.
\end{itemize}

\subsection{CMakeLists file}
SWIG can be incorporated in many build systems, in this case CMake, which can detect the SWIG executable and find the required packages and libraries for linking in order to build shared libraries. Using a single cross platform file (CMakeLists.txt) and two simple commands: cmake and make, CMake generates native build files such as makefiles, nmake files and Visual Studio projects which call SWIG and compile the generated C++ files into shared objects (.so for UNIX or .pyd for Windows).

\subsection{Python Testprogram}
The testprogram (available as a skript and a notebook) allows the setup of materials, sevice, Scenario etc., extracts metadata from an XML file, calls the C++ function that calculates the required results, displays them and  stores them in addition to the metadata in a HDF5 File.\\

To run the program, the user can clone the project from Github, create a build directory and run the commands: cmake.. and make, this will generate all the necessary files to compile and link in the same folder. Afterwards the testprogram must be brought and run through: python project.py. 

Moreover, it was seeked to make the project as flexible as possible, or in other words compatible with different versions of Python as well as realisable on variable operating systems.
\section{Compatibility with Python3}
SWIG is compatible with the different versions of Python (Python 2.7 and Python 3.x). To choose a specific version, all that has to be done is to indicate it in the command line of compiling with cmake, which will find the corresponding packages and libraries and use them to compile and link. As an example, here are the both possibilities to compile on Linux with:
\begin{itemize}
\item \textbf{Python 2.7} 
cmake .. -DPYTHON\textunderscore LIBRARIES=/usr/lib/python2.7
make
python project.py
\item \textbf{Python 3} 
cmake .. -DPYTHON\textunderscore LIBRARIES=/usr/lib/python3
make
python3 project.py
\end{itemize}

\section{SWIG on Windows}
Another approach is available on Windows: building the extension module using a configuration file (conventionally called setup.py), it creates an extension module object using the source code files generated by swig, in addition to the original C++ source and compiles it into a shared object file or DLL (.pyd on Windows), which can be called in the testprogram. 


