\chapter{Theoretical Background}
\label{chapter:theory}
In this project,a high performing programming language was required for the simulation because of the large size of data and the elaborate numerical treatments. C++ fullfills these criteria not to mention that it is available everywhere and reasonably well standardized. Nevertheless it has some drawbacks: First of all it is not interactive and implementing it for user-interfaces can be quite complex(see Tab.~\ref{table:settings}).\\
That is why it was opted for Python for the interface which brings flexibilty interactivity and simplicity.
To sum it up, the goal is to create a common interface that can be provided with different C++ libraries, these will be wrapped in Python modules and loaded dynamically during the execution of the program.\\



\begin{table}[htb]
\centering
\begin{tabular}{|cc|}
\hline
\textbf{C++} & \textbf{Python}\\
\hline
\textbf{Advantages} & \textbf{Advantages} \\
High performance and speed & Flexibility (fast edit-build-debug cycle)
\\useful for intensive tasks & Interactivity\\

Parallelization techniques & (create, change, view objects at runtime) \\

\textbf{Drawbacks} & \textbf{Drawbacks} \\

Non-interactive & relatively slow\\
Writing user-interfaces & Limitations with memory intensive tasks\\
is complex &  Limitations with database access\\
\hline
\end{tabular}
\caption{Characterictics of C++ and Python.\label{table:settings}}

\end{table}